% @Author: Macsnow
% @Date:   2017-04-24 20:19:59
% @Last Modified by:   Macsnow
% @Last Modified time: 2017-04-24 20:20:02

\begin{enumerate}
    \item[3.2] 在题1中定义的基本表上完成以下查询:
    \begin{enumerate}[(1)]
        \item 查询选修课程“CS-02”的学生学号、成绩。
        \item 查询选修课程“EE-01”的女学生姓名。
        \item 查询不选修课程“CS-02”的学生姓名。
        \item 查询身高高于“王涛”同学的男生学号、姓名及年龄。
        \item 查询选修课程“CS-01”的学生中成绩最高的学生学号。
        \item 查询学生姓名及其所选修课程的课程号、学号和成绩。
        \item 查询平均成绩超过80分的学生姓名和平均成绩。
        \item 查询选修三门及以上课程(包括三门)的学生已获得的学分数,并按学号进行升序排列。
    \end{enumerate}
    
    \textbf{答:}
    
    \begin{enumerate}[(1)]
        \item \textbf{select} S\#, Grade \textbf{from} SC \textbf{where} C\#='CS-02';
        \item \textbf{select} Sname \textbf{from} Student \textbf{where} Sex='女' \textbf{and} S\# \textbf{in} (\textbf{select} S\# \textbf{from} SC \textbf{where} C\#='EE-01');
        \item \textbf{select} Sname \textbf{from} Student \textbf{where} S\# \textbf{in} (\textbf{select} S\# \textbf{from} SC \textbf{where} C\#<>'CS-02');
        \item \textbf{select} S\#, Sname \textbf{datediff(yy, Bdate, getdate()} \textbf{where} Height > (\textbf{select} Height \textbf{from} Student \textbf{where} Sname='王涛');
        \item \textbf{select} S\# \textbf{from} SC \textbf{where} C\#='CS-01' \textbf{group by} C\# \textbf{having} Grade=\textbf{max}(Grade);
        \item \textbf{select} Sname, Course.C\#, Credit, Grade \textbf{from} Student, Course, SC \textbf{where} Student.S\#=SC.S\# \textbf{and} Course.C\#=SC.C\#;
        \item \textbf{select} Sname, \textbf{avg}(Grade) \textbf{from} Student \textbf{inner join} SC \textbf{on} Student.S\#=SC.S\# \textbf{group by} S\# \textbf{having} \textbf{avg}(Grade)>80;
        \item \textbf{select} \textbf{sum}(Credit) \textbf{from} SC, Course \textbf{where} SC.C\#=Course.C\# \textbf{group by} S\# \textbf{having} \textbf{count}(C\#)=>3 \textbf{order by} S\# ASC;
    \end{enumerate}

    \item[3.3] 分别在Student和Course表中加入记录(‘01032005’,‘刘静’,‘男’,‘1983-12-10’,1.75,‘西14舍312’)及(‘CS-03’,‘离散数学’,64,4,‘陈建明’)。
    
    \textbf{答:}
    
    \textbf{insert into} Student(S\#, Sname, Sex, Bdate, Height, Dorm) \textbf{values}(‘01032005’, ‘刘静’, ‘男’, ‘1983-12-10’,1.75,‘西14舍312’);
    
    \textbf{insert into} Course(C\#, Cname, Period, Credit, Teacher) \textbf{values}(‘CS-03’,‘离散数学’,64,4,‘陈建明’);
    
    \item[3.4]
    将Student表中已修学分数大于110的学生记录删除。
    
    \textbf{答:}
    
    \textbf{delete from} Student \textbf{where} S\# \textbf{in} (\textbf{select} S\# \textbf{from} SC \textbf{where} Course.C\#=SC.C\# \textbf{group by} S\# \textbf{having sum}(Credit)>110));
    
    \item[3.5]
    将“张明”老师负责的“信号与系统”课程的学时数调整为56,同时增加一个学分。
    
    \textbf{答:}
    
    \textbf{update} Course \textbf{set} Period=56 \textbf{and} Credit = Credit + 1 \textbf{where} Cname=‘信号与系统’ \textbf{and} Teacher=‘张明’;
    
    \item[3.7]
    在题3.1定义的基本表上建立如下视图:
    \begin{enumerate}
        \item 所有居住在“西18”舍的男生视图,包括学号、姓名、出生日期、身高等属性。
        \item “张明”老师所开设课程情况的视图,包括课程编号、课程名称、平均成绩等属性。
        \item 所有选修了“人工智能”课程的学生视图,包括学号、姓名、成绩等属性。
    \end{enumerate}
    
    \textbf{答:}
    
    \begin{enumerate}[(1)]
        \item \textbf{create view} S\_male\_w18 \textbf{as select} S\#, Sname, Bdate, Height \textbf{from} Student \textbf{where} Dorm=‘西18’ \textbf{with check option};
        
        \item \textbf{crate view} Z\_Course \textbf{as select} C\#, Cname, \textbf{avg}(Grade) \textbf{from} Courese, SC \textbf{where} Course.C\#=SC.C\# \textbf{and} Teacher=‘张明’ \textbf{group by} C\# \textbf{with check option};
        
        \item \textbf{create view} AI\_Course \textbf{select} S\#, Sname, Grade \textbf{from} Student, SC \textbf{where} Student.S\#=SC.S\# \textbf{and} C\# \textbf{in} (\textbf{select} C\# \textbf{from} Course \textbf{where} Cname=‘人工智能’) \textbf{with check option};
    \end{enumerate}
\end{enumerate}